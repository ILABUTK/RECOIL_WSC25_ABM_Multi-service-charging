\documentclass{article}

\begin{document}

\title{Change it to WSC'25 template}
\author{Your Name}
\date{\today}

\maketitle

As the adoption of Electric Vehicles (EVs) and Hydrogen Fuel-Cell Vehicles (HFCVs) grows, understanding how dynamic pricing strategies influence charging and refueling behaviors becomes crucial for optimizing local energy markets. A new type of charging station is proposed with integrating solar energy and a comprehensive hydrogen storage system, including a hydrogen tank, electrolyzer, and fuel cell. This study employs a simulation-based approach to analyze the impact of electricity and hydrogen price fluctuations on vehicle decision-making. Here, the charging station operates as the leader, while EVs and HFCVs act as followers. The charging station dynamically adjusts prices based on demand and supply conditions, while vehicles respond by optimizing their charging and refueling schedules. The simulation framework captures real-time interactions, considering factors such as price elasticity, station utilization, and vehicle waiting times. A series of experiments are conducted to investigate key factors affecting system dynamics. From the charging station side, we analyze the charging station’s electricity and hydrogen storage management, assessing how past demand influences future pricing and resource allocation strategies. Besides, we examine the price sensitivity of EVs and HFCVs under different demand levels, identifying how vehicles adjust their charging and refueling strategies in response to cost variations. Additionally, we investigate the effects of time-of-use pricing and peak-hour surcharges on vehicle distribution, station congestion, and energy utilization.
Results demonstrate 

\section{Introduction}
This is a simple LaTeX document with minimal setup.

\section{Content}
You can write your content here using standard LaTeX syntax.

\section{Conclusion}
This concludes our minimalist document.

\end{document}
